%%
\vspace{-2mm}
\section{Privacy Regulations}
\label{sec:regulations}

\vspace{-3mm}
\para{Regulatory Actions in the US.}
\label{sec:us-regulations}
% \vspace{-2mm}
%%
In the US, states enact their own privacy legislation and only a few narrow privacy laws exist at the federal level, notably for children’s personal data (COPPA)~\cite{ChildrensOnlinePrivacy2013}, protected health (HIPAA)~\cite{rightsocrHIPAAPrivacyRule2008}, and personal financial (Gramm–Leach–Bliley Act)~\cite{GrammLeachBlileyAct2013} information. Thus, apart from mandatory provisions, the notice and choice principle---generally implemented via privacy policies---governs what a recipient of personal information can do with it~\cite{zimmeckInformationPrivacyLaw2013}. 
%
Research has surveyed this principle~\cite{schaubDesignSpaceEffective2015} and has shown that it suffers from a lack of regulatory enforcement~\cite{cranorNecessaryNotSufficient2012}, vagueness and ambiguity of notices~\cite{reidenbergAmbiguityPrivacyPolicies2016}, unusable choice implementations~\cite{habibItsScavengerHunt2020}, and nudging and inconvenience factors~\cite{oconnorUnclearInconspicuousRight2021}.

Under its jurisdiction, the FTC can consider privacy policies that misrepresent a business's data handling practices as unfair or deceptive, affecting commerce per 15 U.S.C. §45(a)(1)~\cite{unitedstates:congress:houseofrepresentatives:officeofthelawrevisioncounselUnfairMethodsCompetition2023}, and has done so in the past~\cite{ftcSnapchatSettlesFTC2014,FTCImposes52019}. With such enforcement actions over the last few decades, the FTC has effectively created a body of common law of privacy~\cite{soloveFTCNewCommon2013}.  
%
Similarly, state attorneys general also increased their regulatory activity based on new state privacy laws: California passed the CCPA in 2020 and CPRA in 2023, soon followed by other states as depicted in~\autoref{fig:timeline}.
%
The systematization and enforcement of privacy laws in the US (and elsewhere) is advancing, though recent changes to the CCPA via the CPRA may negatively impact the usability, scope, and visibility of the right to opt-out of sale~\cite{charatanTwoStepsForward2024}.

%%
\para{Regulatory Actions in the EU.}
\label{sec:eu-regulations}
% \vspace{-2mm}
%%
Several EU states established the first data protection laws in late 1970s~\cite{LoiNdeg78171978,GermanDP-1977,NorwayDP-1978}, followed by the EU Data Protection Directive in 1995~\cite{Directive199546EC}, and the GDPR applicable to all EU member states in 2018~\cite{Regulation2016679}.  
%
Personal data transfers from the EU to the US are currently regulated by the EU–US Data Privacy Framework~\cite{EU-US-DP-2023} that replaced prior invalidated frameworks~\cite{PrivacyShield-2016,Schrems-II,SafeHarbor-2000,Schrems-I}.
%
The ePrivacy Directive (2002, amended in 2009) requires in its Article 5(3) a valid user’s consent before \textit{``storing of information, or the gaining of access to information already stored, in the terminal equipment’’}~\cite{Directive2002582002,Directive20091362009}. The GDPR re-defined this notion by setting higher-level legal requirements~\cite{santosAreCookieBanners2020}. As efforts to update the ePrivacy Directive into a Regulation have not reached a consensus so far~\cite{europeancommissionProposalEPrivacyRegulation2024}, EU regulators continuously update their national laws and compliance guidelines to further interpret and implement the ePrivacy Directive~\cite{Bielova2024-zr}.

Therefore, Article 5(3) of the ePrivacy Directive was interpreted in different ways to (a) require consent before cookies are set, read, or sent to third-parties, (b) establish that consent is not required for all tracking technologies if their use is \textit{``strictly necessary’’} (\eg{}, for load balancing) or needed for \textit{``enabling the communication’’}, and (c) cover various types of devices (such as mobile and IoT) and technologies (tracking pixels, link decoration)~\cite{Guidelines22023}. 
%
EU regulators have also been actively investigating tracking technologies, consent, and malpractices. In the Planet49 case, the highest court in the EU (CJEU) established legal precedent by declaring pre-ticked boxes in consent design interfaces illegal~\cite{CJEUC67317}. Similarly, the French Data Protection Authority found that consent banners must offer a reject option on the first layer~\cite{ClosureInjunctionIssued2023,CNILFranceSAN2021024}, and companies were fined for setting cookies prior to consent~\cite{CNILFranceSAN2020012,DeliberationSAN20200127} (for more decisions, see GDPRhub~\cite{GDPRhub}).

In recent years, the EU Commission %’s Cookie Pledge 
tried to establish simpler consent rules~\cite{europeancommissionCookiePledgeEuropean2023}, and EU laws, such as the Digital Markets Act~\cite{RegulationEU20222022} and Digital Services Act~\cite{RegulationEU20222022a} have set up additional rules on valid consent for major companies (defined as \textit{``gatekeepers’’}) and for dark patterns and advertising on online platforms, respectively.

%%
% \vspace{-1mm} % submission
\para{Policy-oriented Solutions.}
\label{sec:policy-solutions}
% \vspace{-2mm}
%%
Several attempts were made at implementing opt out and consent signals for users to communicate their privacy preferences to services. However, the adoption and enforcement of such signals by both senders and recipients is an unresolved coordination problem~\cite{hilsPrivacyPreferenceSignals2021}.

\noindent \textbf{\textit{Platform for Privacy Preferences Project (P3P)}}~\cite{reaglePlatformPrivacyPreferences1999,cranorPlatformPrivacyPreferences2002,cranorPlatformPrivacyPreferences2006} enabled websites to communicate their privacy practices to users in a standardized and fine-grained format. Nonetheless, its utility was limited by the low number of sites that were adopting it~\cite{cranorUseP3PUser2002} and those implementing it correctly and transparently~\cite{cranorAnalysisP3PDeployment2003, leonTokenAttemptMisrepresentation2010}.

\noindent \textbf{\textit{Do Not Track (DNT)}}~\cite{fieldingTrackingPreferenceExpression2019}, developed in 2009 as a binary opt out signal, also saw its adoption to remain low. Indeed, COPPA, which influenced the design of DNT, only requires online services to say whether or not they respect it~\cite{CaliforniaCodeBPC2003}.

\noindent \textbf{\textit{Global Privacy Control (GPC)}}~\cite{zimmeckGlobalPrivacyControl2024} can be viewed as a successor to DNT. While people find GPC useful and usable, adoption is slow~\cite{zimmeckUsabilityEnforceabilityGlobal2023,zimmeckGeneralizableActivePrivacy2024}, despite GPC compliance being required in California (2021)~\cite{archive-attorneygeneralbecerra[@agbecerra]CCPARequiresBusinesses2021,bontaAttorneyGeneralBonta2022} and Colorado (2024)~\cite{coloradodepartmentoflawUniversalOptOutShortlist2024}. Whether GPC can be applicable in the EU within ePrivacy and GDPR context, is still an open discussion~\cite{berjonGPCGDPR2021}.

Policy-oriented protocols and frameworks remain in early stages, as evidenced by the Data Rights Protocol~\cite{consumerreportsDataRightsProtocol2024} and industry consent frameworks~\cite{iabtechlabGlobalPrivacyPlatform2024}.
