%%
\section{Introduction}
\label{sec:introduction}
\vspace{-3mm}

%%
Online users access a variety of free content and services on the web, which are largely funded through online advertising.
%
In turn, advertising is heavily dependent on monitoring users' online activities for various purposes 
%
such as analytics, personalized (re-)targeting, and conversion tracking.
%
To realize these objectives, user tracking has become a pervasive part of the web.
%
Online advertising in the US alone is set to exceed \$400 billion in 2025~\cite{adSpendEmarketer2025}.

%%
Ever since the introduction of cookies on the web in mid-1990s~\cite{cookieIntroduction}, web tracking has evolved into a significantly more pervasive  and sophisticated practice.
%
There has been an increase in prevalence of third-party trackers---with around 92\% of webpages today embedding at least one tracker~\cite{urban2024WebAlmanac2024}.
%
Moreover, user tracking and profiling often involves collection of user's personal details (such as name, email, and location), device characteristics (such as device model and operating system), browsing history, and behavioral signals (such as time spent on a page and performed interactions).
%
As a result, web tracking has become an active area in online privacy research.


%%
Researchers have conducted numerous studies to examine the evolution of web tracking mechanisms, browser developments, and regulatory compliance. 
%
Yet, despite this considerable body of work, major findings remain scattered across many disparate studies.
%%
Furthermore, as privacy defenses improve in browsers, trackers continually adapt with new evasion techniques~\cite{narayanan2018web}.
%
The result is an ever-shifting technical landscape of tracking techniques.
%
Regulations often govern tracking practices and ensure that browsers provide necessary protections to safeguard user privacy.
%
Although these regulatory changes have had a more gradual impact than browser-based technical interventions, together they have continued to reshape the ecosystem. 
%
Today, web tracking is undergoing a transformative change due to the introduction of privacy-enhancing  protections in major web browsers and evolving regulatory frameworks.
%
Recent advancements in online advertising comprises the introduction of privacy-preserving paradigms~\cite{privacySandbox} and adoption of generative AI on the web~\cite{perplexity2025, operator2025, builtin-ai}~\cite{chavezNewPathPrivacy2024}.
%
In the light of these shifts, it is important and timely to comprehensively and systematically study emerging trends in the evolving tracking landscape to identify crucial research gaps.
%
Thus, the research community can clearly benefit from a unified resource that consolidates and systematizes the state of knowledge, helping researchers to make meaningful contributions to the field and ensure a structured approach at addressing new privacy issues.

%%
To this end, in this SoK, we synthesize the disparate lines of research and practices in web tracking---spanning across technical mechanisms, browser mitigations, as well as regulatory changes---to systematically provide an overview of the current state of web tracking. 
%
We scope this work to how the data is \textit{collected} about users, not how that data might then be \textit{used}. 
%
This unified perspective enables a critical reflection on how far the community has come and where it should head next in terms of research directions.
%
Our contributions are as follows:
\vspace{-1mm}

\begin{itemize}
    \item We systematically organize the extensive body of research on web tracking, providing a consolidated knowledge base of advances in the field, highlighting evolving trends, bridging emerging but related tracking mechanisms and identifying gaps in the field.

    \item We provide an overview of major browser-based anti-tracking interventions and relevant regulatory frameworks across the EU and the US to assess how they have altered the ecosystem over the years.

    \item We identify key open challenges and promising future directions in the domain of web tracking for the community to address in the coming years.
\end{itemize}