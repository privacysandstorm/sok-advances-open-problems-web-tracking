%%
\section{Methodology}
\label{sec:methodology}

Online tracking has a vast literature, comprising numerous research studies published over the last few decades.
%
As a result, we first carry out a literature survey to identify all papers related to web tracking published in the last 20 years (2005 onward) at any of the seven top web security and privacy venues---IEEE S\&P, USENIX Security, ACM CCS, NDSS, ACM IMC, PETS, and WWW.
%
A total of 200+ research papers were identified. 
%
Each paper was assigned one or more topics related to web tracking based on the abstract of the paper.
%
The assignment of topics was jointly performed by two researchers following Clarke and Braun’s~\cite{clarke2013successful} thematic analysis approach.
%
A total of 84 topical themes were identified, with the top 15 (by number of papers) being tracking measurement, third-party tracking, browser fingerprinting, cookie consent, cookies, profiling, user studies in tracking, tracking in mobile, ad blocking, regulation compliance, JavaScript tracking, browser extension fingerprinting, advertising and tracking detection, and privacy.
%
We will make the thematic organization of papers public upon acceptance.
%
We used our domain expertise to structure the SoK around these prominent themes as outlined in the rest of this paper. 


%%
Overall, we first provide some background to explain key concepts related to web tracking (\autoref{sec:background}). 
%
Next, we present a unified threat model of web tracking in \autoref{sec:threat-model}. Sections \ref{sec:stateful-tracking} and \ref{sec:stateless-tracking} discuss evolution of stateful and stateless tracking, respectively. 
%
This is followed by a discussion related to cross-device tracking (\autoref{sec:cross-device}) and measurement methodologies used in web tracking studies (\autoref{sec:measurement-methodologies}). 
%
Next, in \autoref{sec:regulations} we compare the evolution of privacy regulations in US vs. EU. 
%
Finally, in \autoref{sec:future-outlook}, we explore future research directions by identifying key open problems that the research community should focus on.
