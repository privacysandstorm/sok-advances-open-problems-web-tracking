%%
\section{Cross-device Tracking}
\label{sec:cross-device}
\textit{Cross-device} tracking~\cite{brookmanCrossDeviceTrackingMeasurement2017,zimmeckPrivacyAnalysisCrossdevice2017} is used to track users across different devices by correlating their identifiers. The technique is used to create profiles of users’ activities that are more comprehensive than those from one device.

\subsection{Types of Cross-device Tracking}

Depending on the certainty with which trackers can correlate activities of a user on different devices, cross-device tracking can be \textit{deterministic} (\eg{}, user logging into the same account) or \textit{probabilistic} (\eg{}, ad networks using IP addresses)~\cite{kimProbabilisticVisitorStitching2017,cottaOffPolicyEvaluationProbabilistic2019}. Traditionally, user’s account information such as username or email addresses have been  used to link or associate browsing activity across different devices. When deterministic identifiers fail, for example, if the user is logged out, probabilistic signals are used such as (a) IP addresses shared by multiple devices belonging to the same user~\cite{diaz-moralesCrossDeviceTrackingMatching2015}, (b) URL browsing patterns since people tend to visit the same websites and apps across devices~\cite{phanCrossDeviceMatching2017}, (c) OS and hardware characteristics~\cite{caoCrossBrowserFingerprintingOS2017}, or (d) typing behavior~\cite{yuanCrossdeviceTrackingIdentification2018}. These features are combined by trackers into \textit{cross-device graphs} to connect different device-specific identifiers like device IDs, cookies, login-based information, and IP addresses with browsing behaviors across devices. These graphs improve the effectiveness of cross-device tracking~\cite{zimmeckPrivacyAnalysisCrossdevice2017,wangGraphTrackGraphbasedCrossDevice2022}.

\subsection{Limitations}

However, probabilistic techniques do not always provide a reliable identifier (\eg{} ISPs dynamically rotate and share public IP addresses across several households). As a result, trackers employ proprietary algorithms to eliminate noise, such as ignoring commercial, private, and proxied IP ranges from cross-device graph computations, or setting fine temporal thresholds for observed identifiers to be considered originating from the same user---for example, at least 10 incoming requests from the same IP address within a 12-hour window.

\subsection{Regulation}

To inform the ad industry of the privacy-invasive nature of cross-device tracking, the FTC held a cross-device tracking workshop in 2015~\cite{ftcCrossDeviceTracking2015}. It also issued warning letters to developers integrating Silverpush, an ad network performing cross-device tracking via inaudible ultrasound signals~\cite{ftcFTCIssuesWarning2016}. Various subsequent studies~\cite{mavroudisPrivacySecurityUltrasound2017,arpPrivacyThreatsUltrasonic2017,matyuninTrackingPrivateBrowsing2018} highlighted the invasiveness of this technique. 

\subsection{Defenses Against Cross-device Tracking}

Tracking protections against deterministic cross-device tracking are inherently limited by users logging into their accounts from different devices. On the other hand, protections against probabilistic cross-device tracking are, principally, the same as against traditional tracking, \eg{}, limiting the disclosure of identifiers or data that could be used to correlate users. 
On mobile devices, techniques have been introduced to intercept, inspect, and block outgoing packets from apps~\cite{shubaNoMoAdsEffectiveEfficient2018}. With respect to the use of inaudible ultrasound signals, efforts have pushed for the standardization of beacons and OS-level APIs to better control access to ultrasound functionality while techniques have been developed to selectively suppress certain frequencies~\cite{mavroudisPrivacySecurityUltrasound2017}.

